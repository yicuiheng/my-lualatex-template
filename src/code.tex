%
% This code is licensed under CC0.
% http://creativecommons.org/publicdomain/zero/1.0/deed
%

\par プログラムのフォントには \code{inconsolata} を使っています.
\par インラインコードの例:  \code{hoge + fuga = piyo}
\par 複数行にまたがるコードの例:

\definecolor{linenocol}{RGB}{100,100,100}

\renewcommand{\theFancyVerbLine}{\color{linenocol}\arabic{FancyVerbLine}}
\newtcblisting{myminted}{%
  listing engine=minted,
  minted language=c,
  listing only,
  breakable,
  enhanced,
  boxrule=0pt,bottomrule=2pt,toprule=2pt,
  sharp corners,
  drop fuzzy shadow,
  title=\hspace*{-1em}\code{</>} \quad bash,
  minted options = {
      linenos,
      breaklines=true,
      breakbefore=.,
      numbersep=2mm
    },
  overlay={%
      \begin{tcbclipinterior}
        \fill[gray!25] (frame.south west) rectangle ([xshift=4mm]frame.north west);
      \end{tcbclipinterior}
    }
}

\begin{myminted}
  fun main() {
      let a = 12
      let b = 30
      a + b
    }
\end{myminted}

\definecolor{theo}{RGB}{232,56,57} % red
\definecolor{lem}{RGB}{2,77,98} % green
\definecolor{def}{RGB}{20,13,74} % blue

\begin{tcolorbox}[enhanced,
    boxrule=0pt,bottomrule=2pt,toprule=2pt,
    colframe=theo,
    sharp corners,
    drop fuzzy shadow,
    title=中間値の定理]
  区間 $[\alpha, \beta]$ で連続な関数 $f(x)$ について,$f(\alpha)$ と $f(\beta)$ の間にある任意の実数 $c$ に対して,
  ある実数 $k \in (\alpha, \beta)$ を $f(k) = c$ を満たすようにとることができる
\end{tcolorbox}

\begin{tcolorbox}[enhanced,
    boxrule=0pt,bottomrule=2pt,toprule=2pt,
    colframe=lem,
    sharp corners,
    drop fuzzy shadow,
    title=中間値の定理]
  区間 $[\alpha, \beta]$ で連続な関数 $f(x)$ について,$f(\alpha)$ と $f(\beta)$ の間にある任意の実数 $c$ に対して,
  ある実数 $k \in (\alpha, \beta)$ を $f(k) = c$ を満たすようにとることができる

\end{tcolorbox}
\begin{tcolorbox}[enhanced,
    boxrule=0pt,bottomrule=2pt,toprule=2pt,
    colframe=def,
    sharp corners,
    drop fuzzy shadow,
    title=中間値の定理]
  区間 $[\alpha, \beta]$ で連続な関数 $f(x)$ について,$f(\alpha)$ と $f(\beta)$ の間にある任意の実数 $c$ に対して,
  ある実数 $k \in (\alpha, \beta)$ を $f(k) = c$ を満たすようにとることができる
\end{tcolorbox}