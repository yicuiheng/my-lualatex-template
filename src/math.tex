\section{数式関連}

\par 定理環境をそれぞれ用意しています.

\begin{definition}{なにかの定義}{nanika}
  なんとかかんとか
\end{definition}
\begin{lemma}{なにかの補題}{nanika}
  あんじゃらかんじゃら
\end{lemma}
\begin{proof}
  \defref{nanika} より即座に成り立つ.
\end{proof}

\begin{theorem}{なにかの定理}{nanika}
  なんじゃらほい
\end{theorem}

\begin{proof}
  これを証明するに先だって次の補題を証明する.
  \begin{lemma}{スーパーなにかの補題}{supernanika}
    スーパーあんじゃらかんじゃら
  \end{lemma}
  \begin{proof}
    内側の証明だよ~
  \end{proof}
  \lemref{nanika} と \lemref{supernanika} より成り立つ.
\end{proof}

\par これらの環境は ぱるちさんのブログ記事
「tcolorboxのお誘い」 \footnote{https://marukunalufd0123.hatenablog.com/entry/2019/03/15/071717} と
「ぼくのかんがえたさいきょうのゼミ用TeXスタイル」 \footnote{https://marukunalufd0123.hatenablog.com/entry/saikyou1}
を参考にして tcolorbox で作っています.

