%
% This code is licensed under CC0.
% http://creativecommons.org/publicdomain/zero/1.0/deed
%

\documentclass[lualatex,a4paper,ja=standard]{bxjsarticle}

\usepackage{style/common}

\begin{document}

%
% This code is licensed under CC0.
% http://creativecommons.org/publicdomain/zero/1.0/deed
%

\begin{titlepage}
  \centering%
  \vspace*{8\baselineskip}
  \rule{\textwidth}{2pt}\vspace*{-\baselineskip}\vspace*{4pt}
  \rule{\textwidth}{1pt}%
  \vspace*{\baselineskip}%
  {\Huge オレオレ $\mathrm{Lua}$ \LaTeX テンプレートの使い方}%
  \vspace*{0.5\baselineskip}%
  \rule{\textwidth}{1pt}\vspace*{-\baselineskip}\vspace*{5pt}
  \rule{\textwidth}{2pt}%
  \vspace*{20\baselineskip}

  {\Large 易 翠衡 (@yicuiheng)} \\
  % {王里国立大学大学院 魔術理論専攻 修士課程}
  {\large \number\year/\number\month/\number\day}
\end{titlepage}


\section{ビルド方法}

\par ビルドには llmk \footnote{https://github.com/wtsnjp/llmk} を使っています.
llmkのおかげで \LaTeX コードのビルド手順を TOML 形式で記述することができます.
その他にもなんか色々とデフォルトの設定をモダンなかんじにしてくれてます.
\LaTeX に詳しくない筆者は取り敢えず llmk を使って \LaTeX 詳しい人が選んだモダンな設定におまかせしています.

\par この プロジェクトをビルドするにはプロジェクトディレクトリで
\begin{program}{bash}{bash}
  $ llmk
\end{program}

を実行してください.llmkが \code{./llmk.toml} に基づいて \code{./main.tex} をコンパイルしてくれます.
うまくいけば \code{./build/main.pdf} が生成されるはずです.

\section{ディレクトリ構成}

\dirtree{%
  .1 {\faFolder\ ./}.
  .2 {\faFolder\ .vscode \DTcomment{VSCode の設定}}.
  .2 {\faFolder\ build \DTcomment{生成物が置かれる場所}}.
  .3 {\faFile\ main.pdf \DTcomment{メインの生成物}}.
  .3 {\dots\dots}.
  .2 {\faFolder\ src \DTcomment{\LaTeX ソースコードを置く場所}}.
  .3 {\faFile\ title.tex \DTcomment{表紙の \LaTeX ソースコード}}.
  .3 {\dots\dots}.
  .2 {\faFolder\ style \DTcomment{スタイルファイルを置く場所}}.
  .3 {\faFile\ common.sty \DTcomment{普遍的に使うマクロやフォントの設定}}.
  .3 {\dots\dots}.
  .2 {\faFile\ llmk.toml \DTcomment{ビルドの設定ファイル}}.
  .2 {\faFile\ main.tex \DTcomment{トップレベルの \LaTeX ソースコード}}.
  .2 {\faFile\ main.bib \DTcomment{参考文献データ}}.
}

\section{フォントなど}

\par 地の文にはUDデジタル教科書体を使っています.
\par あのイーハトーヴォのすきとおった風、夏でも底に冷たさをもつ青いそら、うつくしい森で飾られたモリーオ市、郊外のぎらぎらひかる草の波。

―――宮沢賢治『ポラーノの広場』より―――

\par The quick brown fox jumps over the lazy dog

\par 数式には $\mathrm{STIX2}$ フォントを使っています.

\begin{align}
  \int_0^{\mathrm{\pi}} \sin x \, \mathrm{d}x = 2 \\
  \sum_{n=1}^\infty \frac{1}{n^2} = \frac{\pi^2}{6}
\end{align}

\section{数式関連}

\par 定理環境をそれぞれ用意しています.

\begin{definition}{なにかの定義}{nanika}
  なんとかかんとか
\end{definition}
\begin{lemma}{なにかの補題}{nanika}
  あんじゃらかんじゃら
\end{lemma}
\par \defref{nanika} より即座に成り立つ.

\begin{theorem}{なにかの定理}{nanika}
  なんじゃらほい
\end{theorem}
% \begin{proof}
\par \lemref{nanika} より成り立つ.
% \end{proof}

こんなかんじ.

\par proof環境も定義したい!

%
% This code is licensed under CC0.
% http://creativecommons.org/publicdomain/zero/1.0/deed
%

\par プログラムのフォントには \code{inconsolata} を使っています.
\par インラインコードの例:  \code{hoge + fuga = piyo}
\par 複数行にまたがるコードの例:

\definecolor{linenocol}{RGB}{100,100,100}

\renewcommand{\theFancyVerbLine}{\color{linenocol}\arabic{FancyVerbLine}}
\newtcblisting{myminted}{%
  listing engine=minted,
  minted language=c,
  listing only,
  breakable,
  enhanced,
  boxrule=0pt,bottomrule=2pt,toprule=2pt,
  sharp corners,
  drop fuzzy shadow,
  title=\hspace*{-1em}\code{</>} \quad bash,
  minted options = {
      linenos,
      breaklines=true,
      breakbefore=.,
      numbersep=2mm
    },
  overlay={%
      \begin{tcbclipinterior}
        \fill[gray!25] (frame.south west) rectangle ([xshift=4mm]frame.north west);
      \end{tcbclipinterior}
    }
}

\begin{myminted}
  fun main() {
      let a = 12
      let b = 30
      a + b
    }
\end{myminted}

\definecolor{theo}{RGB}{232,56,57} % red
\definecolor{lem}{RGB}{2,77,98} % green
\definecolor{def}{RGB}{20,13,74} % blue

\begin{tcolorbox}[enhanced,
    boxrule=0pt,bottomrule=2pt,toprule=2pt,
    colframe=theo,
    sharp corners,
    drop fuzzy shadow,
    title=中間値の定理]
  区間 $[\alpha, \beta]$ で連続な関数 $f(x)$ について,$f(\alpha)$ と $f(\beta)$ の間にある任意の実数 $c$ に対して,
  ある実数 $k \in (\alpha, \beta)$ を $f(k) = c$ を満たすようにとることができる
\end{tcolorbox}

\begin{tcolorbox}[enhanced,
    boxrule=0pt,bottomrule=2pt,toprule=2pt,
    colframe=lem,
    sharp corners,
    drop fuzzy shadow,
    title=中間値の定理]
  区間 $[\alpha, \beta]$ で連続な関数 $f(x)$ について,$f(\alpha)$ と $f(\beta)$ の間にある任意の実数 $c$ に対して,
  ある実数 $k \in (\alpha, \beta)$ を $f(k) = c$ を満たすようにとることができる

\end{tcolorbox}
\begin{tcolorbox}[enhanced,
    boxrule=0pt,bottomrule=2pt,toprule=2pt,
    colframe=def,
    sharp corners,
    drop fuzzy shadow,
    title=中間値の定理]
  区間 $[\alpha, \beta]$ で連続な関数 $f(x)$ について,$f(\alpha)$ と $f(\beta)$ の間にある任意の実数 $c$ に対して,
  ある実数 $k \in (\alpha, \beta)$ を $f(k) = c$ を満たすようにとることができる
\end{tcolorbox}
\section{参考文献関連}

\par 参考文献には biber \footnote{http://biblatex-biber.sourceforge.net/} を使っています.
ようわからんけど bibtex の代替だそうで,ナウくて柔軟性があるらしいので採用しています.


\renewcommand{\bibname}{参考文献}
\printbibliography

\end{document}
